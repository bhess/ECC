\documentclass[11pt,english]{article}
\usepackage[latin9]{inputenc}
\usepackage{geometry}
\usepackage{subfig}	%%for easy placement of multiple pictures inside one figure
\geometry{verbose,letterpaper,tmargin=1in,bmargin=1in,lmargin=1.3in,rmargin=1.3in}
\usepackage{graphicx}
\usepackage{amssymb}
\makeatletter

%%%%%%%%%%%%%%%%%%%%%%%%%%%%%% LyX specific LaTeX commands.
\newcommand{\noun}[1]{\textsc{#1}}
%% Bold symbol macro for standard LaTeX users
\providecommand{\boldsymbol}[1]{\mbox{\boldmath $#1$}}

%%%%%%%%%%%%%%%%%%%%%%%%%%%%%% User specified LaTeX commands.
\usepackage{color,calc}
\definecolor{shade}{gray}{0.984}
\newenvironment{myFramedBox}[1][]%
        {
        %\setlength{\fboxsep}{-\fboxrule}
        % \footnotesize\normalfont\ttfamily\raggedright
        \setlength{\rightmargin}{\leftmargin}
        \setlength{\itemsep}{-12pt}
        \setlength{\parsep}{20pt}
        \begin{lrbox}{\@tempboxa}%
        \begin{minipage}{\linewidth-2\fboxsep}
        }%
        {
        \end{minipage}%
        \end{lrbox}%
        \fcolorbox{black}{shade}{\usebox{\@tempboxa}}\newline\newline
        }%

\usepackage{babel}
\makeatother

\setlength{\textwidth}{6.50in}      % was 6.00
\setlength{\evensidemargin}{0.05in} % was 0.25
\setlength{\oddsidemargin}{0.05in}  % was 0.25
\setlength{\textheight}{8.80in}     % was 8.5 or 9.0
\setlength{\topmargin}{-0.7in}      % was -0.5
\setlength{\parskip}{2.0mm}         % was 2
\setlength{\baselineskip}{1.7\baselineskip}

\newtheorem{theorem}{Theorem}
\newtheorem{lemma}[theorem]{Lemma}
\newenvironment{proof}{{\noindent \em Proof:~}}{\hfill{\hfill$\Box$}}

\newcommand{\complexityclass}[1]{{\bf #1}}
\newcommand{\NP}{\complexityclass{NP}}
\newcommand{\RE}{\complexityclass{RE}}

\begin{document}

\def\thepage{}

\title{Elliptic Curve Cryptography\\Final report for a project in computer security}


\author{
   Gadi Aleksandrowicz\thanks{
      Dept.\ of Computer Science,
      The Technion---Israel Institute of Technology,
      Haifa~32000, Israel.
      E-mail: {\tt gadial@cs.technion.ac.il}
   } \and
   Basil Hess
}

\date{}

\maketitle

\section{Introduction}
An \emph{Elliptic Curve} can be roughly described as the set of solutions to an equation of the form $y^2=x^3+ax+b$ over some field (e.g. $\mathbb{C}, 
\mathbb{R},\mathbb{Q}$ or some finite field $\mathbb{F}_{p^n}$). The importence of elliptic curves stems from their rich structure: there is a rather simple
addition law definable on elliptic curves which makes them into an abelian group. Studying the emerging structure of elliptic curves over various fields has been a
major theme in the mathematics of the 20th centaury, and elliptic curves were connected to many famous problems and results, most notably the proof of Fermat's
last theorem\marginpar{Give some reference, to a good survey if possible}.

In the late 70's, \emph{Public-Key Cryptography} systems were first publicly described, changing the face of cryptography. Many of the purposed systems, such
as the Diffie-Hellman key exchange system and the ElGamal encryption system\marginpar{Reference to both papers}, were based on arithmetic in the group $\mathbb{Z}_p$, but in theory could
be implemented in other groups as well. Groups based on elliptic curves were a good choice because of their well-developed theory and their high variety. In 1985,
both Neal Koblitz and Victor S. Miller \marginpar{Ref...} suggested public key cryptosystems based on elliptic curves.

Our goal in the project has been to implement an efficient public key cryptosystem based on elliptic curves ``from scratch'', relying only on large-number arithmetic
libraries. In particular, we have implemented all the elliptic-curve related calculations, and additional related algorithms.

\section{Elliptic Curves}
\subsection{Definition}
Most generally, and elliptic curve $E$ over a field $K$ can be described as the subset of $K\times K$ satisfying the equation $$y^2+a_1xy+a_3y=x^3+a_2x^2+a_4x+a_6$$ 
for a given $a_1,a_2,a_3,a_4,a_5,a_6\in K$, along with another special point ``at infinity'' $\mathcal{O}$. An additional demand is that the curve be ``smooth'',
which means that the partial derivatives of the curve has no common zeros. This can be reduced to checking that some invariant value $\Delta$ (the \emph{discriminant} of the curve)
which is calculated from the coefficients is not zero\marginpar{Add a picture}. Because the nature of the field $K$ is highly relevant to the structure of $E$,
it is common to write $E/K$ indicating the curve $E$ (usually given by the equation) iis defined over the field $K$.

Depending on the characteristic of $K$, the above equation can be simplified. There are three cases to consider:
\begin{enumerate}
  \item When $charK\ne 2,3$ the equation can be simplified to $y^2=x^3+ax+b$ with $a,b\in K$.
  \item When $charK = 2$ and $a_1 \ne 0$, the equation can be simplified to $y^2+xy=x^3+ax^2+b$ with $a,b\in K$. This curve is said to be \emph{non-supersingular}. If $a_1 = 0$, the equation can be simplified to $y^2+cy=x^3+ax+b$ with $a,b,c\in K$. This curve is said to be \emph{supersingular}.
  \item When $charK = 3$ and $a_1^2\ne -a_2$, the equation can be simplified to $y^2=x^3+ax^2+b$ with $a,b\in K$. This curve is said to be \emph{non-supersingular}. If $a_1^2=-a_2$, the equation can be simplified to $y^2=x^3+ax+b$ with $a,b\in K$. This curve is said to be \emph{supsersingular}.
\end{enumerate}

\subsection{Motivating Example: $K=\mathbb{C}$}
A \emph{lattice} in the complex plane $\mathbb{C}$ is a set of the form $L=\{n\omega_1+m\omega_2|n,m\in\mathbb{Z}\}$ such that $\omega_1,\omega_2\in\mathbb{C}$ are
not colinear. Lattices can be used to generalize the notion of ``periodic function'' to the complex plane: A meromorphic function $f$ is called \emph{elliptic} with respect to the lattice
$L$ if $f(z+l)=f(z)$ for all $l\in L$ (equivalently, $f(z+\omega_1)=f(z+\omega_2)=f(z)$). Thus it suffices to know the values of $f$ of the \emph{fundemental parallelogram} $\Pi=\{a\omega_1+b\omega_2|a,b\in[0,1]\}$
in order to know its values on the complex plane.

In $\mathbb{R}$, the periodic function $\sin(x)$ is ``special'' in the sense that along with its derivative, $\cos(x)$. it can be used to represent all
the periodic functions (via Fourier series). There is a parallel ``special'' function for elliptic functions, which in some regards is ``even better'' than its real counterpart. 
Given a lattice $L$, the \emph{Weierstrass $\wp$-function} for this lattice is defined as $$\wp(z)=\frac{1}{z^2}\sum_{0\ne l\in L}\left(\frac{1}{(z-l)^2}-\frac{1}{l^2}\right)$$
$\wp$ has the property that every elliptic function (in respect to a given lattice $L$) can be written as a \emph{rational} function in $\wp$ and $\wp'$ (as opposed to the
infinite sum of $\sin$ and $\cos$). Thus $\wp$ is the most significant example of an elliptic function.

Let $E/\mathbb{C}$ be an elliptic curve given by $y^2=x^3+ax+b$. A lattice $L$ can be found such that the corresponding Weierstrass function satisfies the differential equation
$(\wp')^2=\wp^3+a\wp+b$. This means that the map $z\mapsto (\wp(z),\wp'(z))$ maps $\mathbb{C}$ to points on the elliptic curve. In particular, $\wp$
has a pole in each lattice point and so $0\mapsto \mathcal{O}$ (this can be made precise by using projective coordinates).. Moreover, it
can be shown that this map is onto, and that $z_1,z_2$ give the same point on the curve if and only if $z_1-z_2\in L$, and so we have a bijection
between $E$  and $\mathbb{C}/L$, the latter set can be thought of as a torus (since $\mathbb{C}/L$ is isomorphic to $\Pi$ where parallel edges are ``glued`` together).




\subsection{The Group Action}


\end{document}
